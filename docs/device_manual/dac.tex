\chapter{Digital to Analog Converter (\tx{DAC} Block)}

The digital to analog converter is a hybrid digital-analog block available on
the \hcdc device~\cite{dac.h}. Figures~\ref{dac:values} and \ref{dac:types}
presents the complete list of digitally settable codes in each DAC block. The
digital to analog converter accepts one digital input that is either read from
memory or a lookup table, and produces one analog output current. 

\noindent\textbf{Location}: Each slice on the \hcdc device contains one
DAC. The DAC may read values from the lookup tables resident on slice 0 
(\tx{source} is \tx{DSRC_LUT0}) or slice 2 (\tx{source} is \tx{DSRC_LUT1})
of the same tile.


\begin{marginfigure}
    \small
    \begin{tabular}{l|l}
      code &values\\
      \hline
      \tx{enable} &\tx{bool_t}\\
      \tx{inv}    &\tx{bool_t}\\
      \tx{range}  &\tx{range_t}\\
      \tx{source} &\tx{dac_src_t}\\
      \tx{const_code} &256\\
      \tx{pmos}\caveat&8\\
      \tx{nmos}&8\\
      \tx{gain_cal}&64\\
    \end{tabular}
    \caption{DAC Values \cite{fu.h}}
    \label{dac:values}
\end{marginfigure}
\begin{marginfigure}
    \small
    \begin{tabular}{l|l}
      code & type \\
      \hline
      \tx{enable} & \static\\
      \tx{inv}    & \static\\
      \tx{range}  &\static\\
      \tx{source} & \static\\
      \tx{const_code} & \dynamic\\
      \tx{pmos}\caveat&\hidden\\
      \tx{nmos}&\hidden\\
      \tx{gain_cal}&\hidden\\
    \end{tabular}
    \caption{DAC Code Types\cite{fu.h}}
    \label{dac:types}
\end{marginfigure}

\section{Block Function}\label{dac:blockfun}

Given a DAC at location \hslice{chip}{tile}{slice}, the behavior of the block
is dictated by the relation presented below. At a high level, the DAC block converts a
digital code to an analog current. The value returned by the function is the
value of the current in $\mu A$. Any behavior not covered in the algorithm
below is undefined:

\begin{algorithmic}
  \If {\tx{enable}}
    \If {\tx{source} = \tx{DSRC_MEM}}

    \State $2 \cdot sign(\tx{inv}) \cdot scale(\tx{range}) \cdot (\tx{const_code}-128)\cdot 128^{-1}$
    \ElsIf {\tx{source} = \tx{DSRC_LUT0}}

    \State $2 \cdot sign(\tx{inv}) \cdot scale(\tx{range}) \cdot (\tx{lut}\hindex{chip}{tile}{slice}{0}-128)\cdot 128^{-1}$
    \ElsIf {\tx{source} = \tx{DSRC_LUT1}}

    \State $2 \cdot sign(\tx{inv}) \cdot scale(\tx{range}) \cdot (\tx{lut}\hindex{chip}{tile}{slice}{2}-128)\cdot 128^{-1}$
    \EndIf
 \EndIf
\end{algorithmic}

The \tx{inv} code determines whether the output signal should be inverted or
not. The \tx{range} code scales the output signal by 1x or 10x. The \tx{range}
code may only be set to \tx{RANGE_MED} or \tx{RANGE_HIGH} for the DAC. Note
that all \static and \dynamic codes are used in the block function.

\subsection{\analoglib Implementation}
The \tx{dac.h} file provides a \tx{computeOutput} function that implements the
block function presented above, given a set of \dynamic and \static codes. The
returned value of this function is normalized (divided by $2 \mu A$).

\section{Calibration}
The DAC block has three hidden codes. The
\tx{pmos} code is always set to $0$. The remaining codes are set in the block's
calibration routine. The DAC is calibrated using the following
algorithm~\cite{dac_calib.cpp}:

\begin{algorithmic}
  \State{\tx{table} = \tx{make}()}
  \For{\tx{nmos} in $0...7$ }
  \For{\tx{gain_cal} in $0...63$ with stride 16 }
    \State {loss = obj(\tx{nmos},\tx{gain_cal})}
    \State {update \tx{table} loss (\tx{nmos},\tx{gain_cal})}
    \EndFor
  \EndFor

  \For{\tx{gain_cal} in $0...63$}
    \State {loss = obj(\tx{table.state.nmos},\tx{gain_cal})}
    \State {update \tx{table} loss (\tx{nmos},\tx{gain_cal})}
  \EndFor
  \State{return \tx{table.state.nmos},\tx{table.state.gain_cal}}
\end{algorithmic}

At a high level, the calibration algorithm iterates over \tx{nmos} and
\tx{gain_cal} codes and computes the loss for each combination of codes. The
loss function is computed using the objective function, \textit{obj}. Objective
functions are evaluated over a collection of 4
test points unless specified otherwise (\tx{const_code} that encodes
0.0,0.8,-0.8,0.5). The expected behavior is computed using the block function
specified in Section~\ref{dac:blockfun}. The DAC block supports three objective functions:
\begin{itemize}
\item\tx{CALIB_MINIMIZE_ERROR}: This objective function minimizes the average
  error between the observed signal and the expected behavior. 
\item\tx{CALIB_MAXIMIZE_DELTA_FIT}: This objective function minimizes the gain
  variance and magnitude bias of the block. The magnitude bias $b$ is computed
  by measuring the signal for test point $0.0$. For the nonzero test
  points, the gain is computed by taking the ratio of the observed to the
  expected value. The gain variance $\sigma^2$ is the computed by taking the
  variance over computed gains. The final returned loss is $min(\sigma,|b|)$.
  \item\tx{CALIB_FAST}: This objective function minimizes the error for test
    point $1.0$. This quickly calibrates the gain to have good gain characteristics.
  \end{itemize}
  
\section{Profiling}

XXX profiling algorithm here XXX
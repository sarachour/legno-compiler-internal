\chapter{Current Copier Block (\tx{fanout} Block)}

The current copier block is an analog block available on the \hcdc
device~\cite{fan.h}. Figures~\ref{fan:values} and \ref{fan:types} presents a
complete summary of the digitally settable codes for the block. The current
copier accepts one analog input and produces three analog outputs, where the
analog outputs are copies of the provided signal.

\noindent\textbf{Location}: Each slice of the \hcdc device contains two current
copier blocks. Given a slice at \hslice{chip}{tile}{slice}, the two current
copiers on the slice are written as \hindex{chip}{tile}{slice}{0} and \hindex{chip}{tile}{slice}{1}.

\begin{marginfigure}
  \small
  \begin{tabular}{l|l}
    code &values\\
    \hline
    \tx{enable}& \tx{bool_t}\\ 
    \tx{range}& \tx{range_t}\\
    \tx{inv[out0Id]}& \tx{bool_t}\\
    \tx{inv[out1Id]}& \tx{bool_t}\\
    \tx{inv[out2Id]}& \tx{bool_t}\\
    \tx{nmos} & 8\\
    \tx{pmos}\caveat & 8\\
    \tx{port_cal[out0Id]}& 64\\
    \tx{port_cal[out1Id]}& 64\\
    \tx{port_cal[out2Id]}& 64\\
  \end{tabular}
  \caption{Fanout Values \cite{fu.h}}
  \label{fan:values}
\end{marginfigure}
  
\begin{marginfigure}
  \small
  \begin{tabular}{l|l}
    code & type \\
    \hline
    \tx{enable}& \static \\ 
    \tx{range}& \static \\
    \tx{inv[out0]}& \static \\
    \tx{inv[out1]}& \static \\
    \tx{inv[out2]}& \static \\
    \tx{nmos} & \hidden \\
    \tx{pmos}\caveat & \hidden \\
    \tx{port_cal[out0]}& \hidden \\
    \tx{port_cal[out1]}& \hidden\\
    \tx{port_cal[out2]}& \hidden \\
  \end{tabular}
  \caption{Fanout Code Types\cite{fu.h}}
  \label{fan:types}
\end{marginfigure}

  
\section{Block Function}\label{fanout:blockfun}
The behavior of output $i$ (\tx{out}$i$) the current copier is dictated by the
relation presented below. We write the analog input as $in$ in the presented
relation. The value returned by the function is the value of the current in $\mu
A$. Any behavior not covered by this algorithm is undefined.

\begin{algorithmic}
  \If{\tx{enable}}
  \State{sign(\tx{inv}[\tx{out}$i$]) $in$} 
  \EndIf
\end{algorithmic}

The \tx{inv} code for the output $i$ determines the whether the copied signal
should be inverted or not. Note that all \static codes, with the exception of
the \tx{range} code, are used in the block function. The \tx{range} code is
used to configure the current limitations of the block.

\subsection{Operating Ranges}

The magnitude of the analog input $in$ must fall within the current limits of
the current copier. These limits are determined by the \tx{range} code. If the
\tx{range} code is set to \tx{RANGE_MED}, the analog input must fall within
$[-2 \mu A, 2 \mu A]$. If the \tx{range} code is set to \tx{RANGE_HIGH}, the
analog input must fall within $[-20 \mu A, 20 \mu A]$

\section{Calibration}\label{fanout:calib}


\begin{algorithmic}
  \State {tbl = \tx{make_table()}}
  \For{\tx{nmos} in 0...7}
    \State {tbl0 = \tx{make_table()}}
    \For{cal0 in 0...64}
      \State {loss = obj(cal0,32,32)}
      \State {tbl0 $\leftarrow$ loss,cal0}
    \EndFor
    \State {tbl1 = \tx{make_table()}}
    \For{cal1 in 0...64}
      \State {loss = obj(tbl0.cal0,cal1,32)}
      \State {tbl1 $\leftarrow$ loss,cal1}
    \EndFor
    \State {tbl2 = \tx{make_table()}}
    \For{\tx{cal2} in 0...64}
      \State {loss = obj(tbl0.cal0,tbl1.cal1,cal2)}
      \State {tbl2 $\leftarrow$ loss,\tx{cal2}}
    \EndFor
    \State{loss = max(tbl2.loss,tbl1.loss,tbl0.loss)}
    \State {tbl $\leftarrow$ loss,(nmos,tbl0.cal0,tbl1.cal1,tbl2.cal2)}
    \EndFor
    \State{return tbl.nmos,tbl.cal0,tbl.cal1,tbl.cal2}
\end{algorithmic}

\section{Profiling}\label{fanout:calib}
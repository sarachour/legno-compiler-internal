\chapter{Analog Block Specification}

\section{Analog Device Architecture}
\section{Digitally Settable Code Types}

\begin{marginfigure}
  \begin{tabular}{c|cc}
    \tx{bool_t} & \tx{true}\\
                & \tx{false}\\
    \tx{range_t} & \tx{RANGE_MED}\\
                & \tx{RANGE_HIGH}\\
                & \tx{RANGE_LOW}\\
  \end{tabular}
  \label{code:types}
\end{marginfigure}

Figure~\ref{code:types} presents a summary of the types of digitally settable
codes. All other digitally settable codes are unsigned integers.

\subsection{Utility Functions}
\noindent\textit{sign}(\tx{bool_t} inverted): The sign function returns a
constant coefficient of $-1$ if inverted is set, otherwise it returns a
coefficient of $1$. Refer to \tx{sign_to_coeff} in \tx{util.h}~\cite{util.h}.

\noindent\textit{scale}(\tx{range_t} range): The range function a constant
coefficient that corresponds to the selected current range. The function returns
1.0 if the range is \tx{RANGE_MED}, 10.0 if the range is \tx{RANGE_HIGH} and 0.1
if the range is \tx{RANGE_LOW}. Refer to \tx{range_to_coeff} in
\tx{util.h}~\cite{util.h}.

\section{Digital to Analog Converter (DAC)}

The digital to analog converter is a hybrid digital-analog block available on
the \hcdc device~\cite{dac.h}. Figure~\ref{dac:codes} presents the complete list
of digitally settable codes in each DAC block. The digital to analog converter
accepts one digital input that is either read from memory or a lookup table, and
produces one analog output current. Figure~\ref{dac:codes} presents a complete summary of codes. 

\noindent\textbf{Location}: Each slice on the \hcdc device contains one
DAC. The DAC may read values from the lookup tables resident on slice 0 
(\tx{source} is \tx{DSRC_LUT0}) or slice 2 (\tx{source} is \tx{DSRC_LUT1})
of the same tile.


\begin{marginfigure}
  \begin{tabular}{c|cc}
    code &values& kind \\
    \hline
    \tx{enable} &\tx{bool_t}& \static\\
    \tx{inv}    &\tx{bool_t}& \static\\
    \tx{range}  &\tx{range_t}&\static\\
    \tx{source} &\tx{dac_src_t}& \static\\
    \tx{const_code} &256& \dynamic\\
    \tx{pmos}\caveat&8&\hidden\\
    \tx{nmos}&8&\hidden\\
    \tx{gain_cal}&64&\hidden\\
  \end{tabular}
  \caption{DAC Parameters\cite{fu.h}}
  \label{dac:codes}
\end{marginfigure}

\subsection{Block Function}

Given a block at location
\hslice{chip}{tile}{slice}, the behavior of the block is dictated by the
following relation:

\begin{algorithmic}
  \If {\tx{enable}}
    \If {\tx{source} = \tx{DSRC_MEM}}

    \State $sign(\tx{inv}) \cdot scale(\tx{range}) \cdot (\tx{const_code}-128)\cdot 128^{-1}$
    \ElsIf {\tx{source} = \tx{DSRC_LUT0}}

    \State $sign(\tx{inv}) \cdot scale(\tx{range}) \cdot (\tx{lut}\hindex{chip}{tile}{slice}{0}-128)\cdot 128^{-1}$
    \ElsIf {\tx{source} = \tx{DSRC_LUT1}}

    \State $sign(\tx{inv}) \cdot scale(\tx{range}) \cdot (\tx{lut}\hindex{chip}{tile}{slice}{2}-128)\cdot 128^{-1}$
    \EndIf
 \EndIf
\end{algorithmic}

Any behavior not covered in the above algorithm is undefined.
The \tx{inv} code determines whether the output signal should be inverted or
not. The \tx{range} code scales the output signal by 1x or 10x. The \tx{range}
code may only be set to \tx{RANGE_MED} or \tx{RANGE_HIGH} for the DAC. Note
that all \static and \dynamic codes are used in the block function.

\subsection{Calibration}
The DAC block has three hidden codes. The
\tx{pmos} code is always set to $0$. The \tx{nmos} and \tx{gain_cal} codes
control the gain of the block. The DAC is calibrated using the following
algorithm:

\begin{algorithmic}
  \For{each item \tx{nmos} in $0...7$ }
    \For{each item \tx{gain_cal} in $0...63$ }
       \State loss = obj(\tx{nmos},\tx{gain_cal})
    \EndFor
  \EndFor
\end{algorithmic}
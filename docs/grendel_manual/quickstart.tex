\chapter{Grendel Quickstart}

The \grendel runtime system (\tx{grendel.py}) enables developers to dispatch
commands to the \hcdc analog device on the fly. The \grendel runtime is used to
execute scripts generated by the \legno compiler as well as calibrate and profile
blocks on the device. 

\subsection{Installation}

The \grendel runtime is a pure python program. To install the python
dependencies, execute the following command:

\begin{lstlisting}
pip install -r packages.list
\end{lstlisting}

Both the \legno compiler and the \grendel runtime work with a \tx{config.py}
file that specifies the relevant output directories and database files. To use
the default configuration, execute the following copy command:

\begin{lstlisting}
cp util/config_local.py util/config.py
\end{lstlisting}

\subsection{Anatomy of a Grendel Script}

The following section of the quickstart guide describes how to execute the
\tx{test/cosfun.grendel} script using the \grendel runtime. The \tx{cosfun}
script configures the analog device to emit a cosine function at output 0 of the
analog device. We describe the different parts of the \tx{cosfun} script below:

Each \grendel script begins with a \tx{micro_reset} command that resets all the
parameters set in the Arduino microcontroller. 

\begin{lstlisting}
micro_reset
\end{lstlisting}

The \grendel script then configures the time and voltage scales of the
oscilloscope so that the waveform fits in the viewport. The \tx{set_volt_range}
commands set the minimum and maximum voltages for channels 0 and 1 of the
scolloscope. The \tx{osc_sim_time} command sets the amount of time the
oscilloscope should record. The \tx{micro_use_osc} command tells the
microcontroller to notify the oscilloscope by generating a trigger signal at the \tx{SDA1} pin. 

\begin{lstlisting}
micro_use_osc
osc_set_volt_range 0 0.102000 1.310000
osc_set_sim_time 2.063e-03
osc_set_volt_range 1 0.102000 1.310000
osc_set_sim_time 2.063e-03
\end{lstlisting}

The next set of operations were initially used when the micro-controller was used
to buffer signals. These commands are currently deprecated.

\begin{lstlisting}
micro_compute_offsets
micro_get_num_adc_samples
micro_get_num_dac_samples
micro_get_time_delta
\end{lstlisting}

The \grendel script then configures the \tx{fanout} and \tx{integ} blocks and
sets the necessary connections to implement the circuit in Figure~\ref{XXX}.

\begin{lstlisting}
micro_use_chip
use_fanout 0 3 0 0  sgn + - + rng m two
get_integ_status 0 3 0
use_integ 0 3 0 sgn + val 0.8359375 rng h m debug
get_integ_status 0 3 1
use_integ 0 3 1 sgn + val 0.0 rng h m debug
mkconn fanout 0 3 0 0 port 0 tile_output 0 3 0 0
mkconn tile_output 0 3 0 0 chip_output 0 3 2
mkconn integ 0 3 0 fanout 0 3 0 0
mkconn integ 0 3 1 integ 0 3 0
mkconn fanout 0 3 0 0 port 1 integ 0 3 1
\end{lstlisting}

With the analog device configured, the \grendel script configures the
oscilloscope to wait for an edge trigger (\tx{osc_setup_trigger}), then executes
the simulation. The waveform is retrieved from the oscilloscope using the
\tx{osc_get_values} command. This command stores the measured waveform in the
\tx{waveform.json} file.

\begin{lstlisting}
micro_get_status
osc_setup_trigger
micro_run
osc_get_values differential 0 1 Pos waveform.json
get_integ_status 0 3 0
get_integ_status 0 3 1
micro_get_status
\end{lstlisting}

After the simulation has finished executing, the \grendel script disables all
the enabled blocks and disables any connections that have been set. 

\begin{lstlisting}
disable fanout 0 3 0 0
disable integ 0 3 0
disable integ 0 3 1
rmconn fanout 0 3 0 0 port 0 tile_output 0 3 0 0
rmconn tile_output 0 3 0 0 chip_output 0 3 2
rmconn integ 0 3 0 fanout 0 3 0 0
rmconn integ 0 3 1 integ 0 3 0
rmconn fanout 0 3 0 0 port 1 integ 0 3 1
\end{lstlisting}

\section{Executing the Grendel Script}

All the blocks in the \tx{cosfun.grendel} script must first be calibrated before
the script can be executed. The following command calibrates any uncalibrated
blocks in the script and stores the \hidden code values computed by the
calibration routine in the \tx{state.db} database:

\begin{lstlisting}
python3 grendel.py --calib-obj min_error calibrate test/cosfun.grendel
\end{lstlisting}

The \tx{--calib-obj min_error} flag
tells the \grendel runtime which objective function the \analoglib library
should use to calibrate the blocks. We use the objective function that minimizes
the error of the block for calibration. 

After the calibration procedure is done, we can then use the grendel runtime to
execute the script. This command runs the simulation on the analog device:

\begin{lstlisting}
python3 grendel.py --calib-obj min_error run test/cosfun.grendel
\end{lstlisting}

THe \tx{--calib-obj} flag specifies which set of \hidden codes to use to
calibrate the block. Recall the \hidden codes are computed during the
calibration process. For this execution, we tell the \grendel runtime to use the
hidden codes that were produced by calibrating the blocks to minimize error.

Figure~\ref{XXX} presents the simulation produced by the analog device (in red)
compared to the reference simulation (in blue). 
\chapter{Batch Execution with Experiment Driver}

The \expdriver batch processing tool automatically finds an executes all \grendel scripts in
the \tx{output} directory. The \expdriver is designed to work with \grendel
scripts generated by the \legno compiler. It therefore relies on the naming
conventions and directory structure used by the \legno compiler.

In addition to batch executing scripts, \expdriver is able to analyze and
produce visualizations for the waveforms generated by the executions. \expdriver
postprocesses the collected signals using the scaling factors produced by the
\legno compiler and compares the recovered signal to a ground-truth reference
signal. The \expdriver is also able to compute the runtime and energy
requirements for each execution.

The \expdriver tracks the state of each experiment in a database located in
\tx{outputs/experiment.db}.

\subsection{Scanning for \grendel Scripts}

We use \expdriver to scan for \grendel scripts using the following command:

\begin{lstlisting}
  python3 exp_driver.py scan
\end{lstlisting}

This command automatically searches for any \grendel scripts in the \tx{outputs}
directory and adds them to the experiment database~\cite{experiment.db}. The
experiment database tracks the execution and analysis results of each
experiment. All scripts start out as \tx{pending} or \tx{ran}. A script is
marked as \tx{ran} if all of the waveforms it is supposed to generate are present.


\subsection{Executing Pending Grendel Scripts}
Next we tell the \expdriver tool to execute any pending experiments.

\begin{lstlisting}
  python3 exp_driver run --calibrate
\end{lstlisting}

The \tx{--calibrate} flag tells the \expdriver command to calibrate any
blocks that require calibration before running the experiments. Note that the
\tx{--calib-obj} flag is automatically inferred from the model used to compile
the script. If the program was scaled by \lscale using a \tx{naive} model, the
\tx{min_error} calibration objective is used for calibration and execution. If the program was
scaled by \lscale using a \tx{partial} or \tx{physical} model, the \tx{max_delta}
calibration objective is used for calibration and execution.

\subsection{Analyzing Grendel Scripts}

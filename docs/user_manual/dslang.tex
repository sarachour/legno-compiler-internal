\chapter{Dynamical System Language}
\label{sec:dslang}
The \legno compiler (\tx{legno.py}) enables developers to compile dynamical
systems down to configurations for the analog hardware. The \legno compiler
requires that dynamical systems be specified in the \tx{dynamical system
  language}, a high level language that supports writing first-order
differential equations.

\section{Example: Cosine Program}

The following dynamical system program implements the cosine function. This
program can be found in \tx{progs/quickstart/cos.py}. In order for \tx{legno.py}
to find the program, the program must be placed in the \tx{progs/} directory.:
\begin{dssnippet}

from dslang.dsprog import DSProg
from dslang.dssim import DSSim,DSInfo

def dsname():
  return "cos"

def dsinfo():
  return DSInfo(name=dsname(), \
                desc="cosine",
                meas="signal",
                units="signal")
  info.nonlinear = False
  return info

def dsprog(prob):
  params = {
    'P0': 1.0,
    'V0' :0.0
  }
  prob.decl_stvar("V","(-P)","{V0}",params)
  prob.decl_stvar("P","V","{P0}",params)
  prob.emit("P","Position")
  prob.interval("P",-1.0,1.0)
  prob.interval("V",-1.0,1.0)
  prob.check()
  return prob

def dssim():
  exp = DSSim('t20')
  exp.set_sim_time(20)
  return exp
  
\end{dssnippet}

\noindent A dynamical system program must define four python functions in order to be used
by the compiler:

\begin{itemize}
\item\textbf{dsname}: This function returns the name of the program. The user
  can later refer to this program by its name during compilation.
\item\textbf{dsinfo}: This function returns detailed information about the
  program. This includes a short description of the program, the name of the
  signal being measured and the units of the signal being measured. Only one
  signal can be measured right now.
\item\textbf{dsprog}: This function returns the dynamical system associated
  with the program. The dynamical system is defined as a collection of state variable and
  function declarations. The dynamical system also contains interval annotations
  for each state variable (this defines the bounds for each variable) and which
  variable is measured. The \tx{check()} function checks to see that the system
  is bounded. 
  \item\textbf{dssim}: This function returns the simulation parameters. This
    program is to be run for 20 simulation units.  
\end{itemize}

\subsection{Breaking Down \tx{dsprog}}
In this dynamical system, the position \tx{P} corresponds to the amplitude of
the cosine function. The $V$ and $P$ variables are internal variables that are
used to model the dynamics of the system. We describe each line of the program
below:

\begin{itemize}
   \item\tx{params = {..}}: This statement creates a parameter map. The
     parameter map associates names with values. This is used to fill in
     parameters when defining the dynamical system. Parameters are referred to
     by their name, enclosed in curly braces (for example \tx{\{V0\}} refers to
     the parameter V0). 
   \item\tx{prob.decl_stvar("V","(-P)","{V0}",params)}: This statement defines a
     state variable $V$, whose dynamics are governed by $V' = -P$. The initial
     value of the state variable is the value of parameter V0. The last argument
     is the parameter map to use.
   \item\tx{prob.decl_stvar("P","V","{P0}",params)}: This statement defines a
     state variable $P$, whose dynamics are governed by $P' = V$. The initial
     value of the state variable is the value of parameter P0. The last argument
     is the parameter map to use. 

   \item\tx{prob.emit("P","Position")}: This statement indicates the expression
     $P$ is of interest, and should be observed. The observation is named
     \tx{Position}.
   \item\tx{prob.interval("P",-1.0,1.0)}: This statement defines the variable
     \tx{P} as falling within the bounds [-1,1]. All state variables must be
       restricted to a user-defined interval in order for the program to pass
       the well formedness check (\tx{check()} invocation).
   \item\tx{prob.interval("V",-1.0,1.0)}: This statement defines the variable
     \tx{V} as falling within the bounds[-1,1].
   \item\tx{prob.check()}: This statement checks that all the variables in the
     dynamical system are bounded. This function must pass (not trigger an error) in
     order for the program to compile successfully.
\end{itemize}



\subsection{Executing the \tx{cos} Dynamical System with a ODE Solver}

The \legno compiler supports digitally simulating dynamical systems for testing
purposes.

\begin{snippet}
  python3 legno.py --subset extended simulate cos --reference
\end{snippet}

The \tx{--reference} argument tells the compiler to perform a digital
simulation and produce reference figures. To profile the runtime of the
reference simulation, add the \tx{--runtime} argument.

Note that the \tx{--subset} argument is used to determine where to write the reference
simulation -- it does not impact the behavior of the simulation. The digital simulator
writes the plots of the state variable trajectories to the following directory:

\begin{snippet}
  legno-compiler/outputs/legno/extended/cos/sim/ref/
\end{snippet}


\section{Creating your own Dynamical System}

The following section walks through how to create a dynamical system named
\tx{deg} that implements exponential decay. First, copy the dynamical system
template to a new file. This file will automatically be picked up by the compiler:

\begin{snippet}
  cp progs/template.py progs/deg.py
\end{snippet}

Next we modify the contents of \tx{test.py} to implement the desired dynamical
system. First we change the name of the program to \tx{decay}:

\begin{dssnippet}
def dsname():
   return "decay"
\end{dssnippet}

Next, we implement the dynamical system by populating the \tx{dsprog} function.
The exponential decay function is implemented as $x' = -k \cdot x$, where $k$
is the decay parameter. The dynamical system takes one additional parameter,
$x(0)$, the initial condition of the system. We are interested in executing the
dynamical system with $k=0.1$ and $x(0) = 10$. We define the following parameter
dictionary:

\begin{dssnippet}
  params = {'k':0.1,'x0':10.0}
\end{dssnippet}

This system has one state variable, $x$. We next define the dynamics of $x$ with
the statement below.


\begin{dssnippet}
  prog.decl_stvar("x","{k}*(-x)","{x0}",params)
\end{dssnippet}

This statement creates a new state variable named \tx{x}, whose derivative is
\tx{k*(-x)} (\tx{k} is a parameter) and initial condition is the
parameter \tx{x0}. Wrapping a variable in curly braces indicates it's a
parameter -- the compiler will look for the parameter in the parameter map
that is passed in. The last argument, \tx{params}, provides the parameter map to
the compiler. 

We are interested in observing this state variable; we tell the compiler we
would like to forward this variable to an external pin for observation with the
\tx{emit} command. This signal will be named \tx{OBS} in any generated plots:

\begin{dssnippet}
  prog.emit('x','OBS')
\end{dssnippet}

We also need to provide an interval range annotation for $x$. This annotation
tells the compiler the range of values $x$ may take on. It is important to set
an appropriate dynamic range for each variable. If this range is smaller
than the actual dynamic range of $x$, it may saturate. If this range is much
larger than the actual dynamic range, then $x$ might be overtaken by noise and
error during execution.

We know that $x$ starts at $10.0$ and decays to $0.0$. It therefore always falls
within $[0,10.0]$. We set the interval of $x$ with the following command:

\begin{dssnippet}
  prog.interval('x',0,10.0)
\end{dssnippet}

The interval of $OBS$ is automatically derived from $x$. The full \tx{dsprog}
function is shown below:

\begin{dssnippet}
def dsprog(prog):
  params = {'k':0.1,'x0':10.0}
  prog.decl_stvar("x","-{k}*x","{x0}",params)
  prog.emit('x','OBS')
  prog.interval('x',0,10.0)
  prog.check()
  return
\end{dssnippet}

We next need to provide the simulation parameters of the system by changing the
\tx{dssim} function. We are fine with executing the simulation for 20 simulation
units, so we can leave the \tx{dssim} function as is:

\begin{dssnippet}
def dssim():
  exp = DSSim('t20')
  exp.set_sim_time(20)
  return exp
\end{dssnippet}

Next, we change the program information to reflect the exponential decay
system. The \tx{decay} system is a linear system that tracks the trajectory of $x$, which is
measured in arbitrary units. 

\begin{dssnippet}
def dsinfo():
  info = DSInfo(dsname(), \
                desc="exponential decay"
                meas="trajectory",
                units="units")
  info.nonlinear = False
  return info
\end{dssnippet}

We can test the dynamical system by digitally simulating with using the command below:

\begin{snippet}
 python3 legno.py --subset extended simulate decay --reference
\end{snippet}

The command will write plots of \tx{x} and \tx{OBS} to the following directory:

\begin{snippet}
 legno-compiler/outputs/legno/extended/decay/sim/ref/
\end{snippet}

\section{Troubleshooting}

\subsection{\tx{handle not in interval} when executing \tx{check()}}

If you get the following error:

\begin{snippet}
  Exception: handle not in interval: :h0
\end{snippet}

This indicates that the compiler could not compute intervals for all the
variables in the dynamical system. Please make sure all state variables and
functions (optionally, usually) are bounded using \tx{interval} commands.

\subsection{\legno cannot find my dynamical system program!}

Make sure that (1) the program is a \tx{.py} file that resides somewhere in the
\tx{legno-compiler/prog} directory (2) the \tx{dssim}, \tx{dsinfo}, \tx{dsname}
and \tx{dsprog} functions are defined in the file.
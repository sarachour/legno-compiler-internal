\chapter{Installation}

The \legno compiler has been tested on OSX and Linux systems. It is a
command-line utility -- all commands in this document should be executed using the command line. 

\section{Setting up the Compilation Environment}
\subsection{Mac OS}


In order to install the dependencies for the \legno compiler, the \tx{brew} package manager (https://brew.sh/). Brew is installed using the following command:

\begin{snippet}
  /usr/bin/ruby -e \
  "\$(curl -fsSL https://raw.githubusercontent.com/Homebrew/install/master/install)"
\end{snippet}

 Once brew is installed, it may be accessed using the command \tx{brew}. Next,
 we should install the required dependences. 

\begin{snippet}
  brew install python3-pip python3 git
\end{snippet}


\subsection{Linux}

First, install the required dependencies

\begin{snippet}
sudo apt-get install -y python3-pip libatlas-base-dev git
\end{snippet}

\section{Setting up the Compilation Toolchain}

First, we need to download the compilation toolchain and make sure we have selected the \tx{master} branch:

\begin{snippet}
  git clone git@github.com:sendyne/legno-compiler.git
  git checkout master
\end{snippet}

This should create a \tx{legno-compiler} directory; inside this directory is the
legno compiler toolchain. The compilation toolchain contains four major components:

\begin{itemize}
\item\textbf{Compiler} (\tx{legno.py} and \tx{legno_runner.py}): The compiler.
  It reads dynamical system programs (in the \tx{prog} directory) and generates
  grendel scripts. These scripts can be executed by the runtime.  
\item\textbf{Runtime} (\tx{grendel.py}): The runtime. It executes grendel
  scripts on the analog device by dispatching commands to the microcontroller.
\item\textbf{Firmware} (\tx{lab_bench} directory): The firmware. This must be
  written to the microcontroller in order for the grendel runtime to work.
\item\textbf{Analyzer} (\tx{exp_driver.py}): This tool analyzes the outputs
  produced by the analog chip. It computes the energy, power, quality (with
  respect to a reference simulation) and runtime of each program.
\end{itemize}

\subsection{Setting up the Compiler}
The compiler is a pure python program, so almost all of the dependences can be
installed using the python package manager, pip. Execute the following command
from the root directory of the \tx{legno-compiler} project to execute all the required packages:

\begin{snippet}
  pip3 install -r packages-legno.list
\end{snippet}

\section{Setting up the HCDCv2 Device}
These steps are only necessary if you wish to execute the generated programs on
the HCDCv2 device. To write the device firmware to the SA100ASY development board, you must install the Arduino IDE. To install the Arduino IDE, follow the following link:

\begin{snippet}
https://www.arduino.cc/en/Main/Software
\end{snippet}

After installing the Arduino IDE, navigate to \tx{Tools/Board} in the menu, and select the \tx{Board Manager}
option. After doing so, search for \tx{Due} and install the \tx{Arduino Sam Boards (32 bits ARM-Cortex M3} package.
We installed 1.6.12.

You also want to install the \tx{DueTimer} package. This can be done by navigating to the 
\tx{Sketch/Include Library/Manage Libraries} option in the Arduino IDE and searching for the \tx{DueTimer} package.
We installed 1.4.7.

\subsection{Installing Arduino-Make}
The firmware is built using a Makefile that extends a specialized set Arduino
Makefiles. The Arduino makefiles are provided in the \tx{arduino-mk} package.

\noindent\textbf{Linux Command}: The following installs the arduino makefiles on linux.
\begin{snippet}
sudo apt-get install -y  android-mk
\end{snippet}

\noindent\textbf{OSX Command}: The following installs the arduino makefiles on OSX:

\begin{snippet}
brew tap sudar/arduino-mk
brew install --HEAD arduino-mk
\end{snippet}

\subsection{Installing the Grendel Runtime Dependences}


The firmware communicates with a runtime which is also written in python.
Execute the following command from the root directory of the \tx{legno-compiler}
project to install all the required packages.

\begin{snippet}
  pip3 install -r packages-grendel.list
\end{snippet}

\subsection{Writing Firmware to the HCDCv2 Device}
After following these steps, it should be possible to flash the HCDCv2 firmware to the analog device. Navigate
to the following directory:
\begin{snippet}
lab_bench/arduino/grendel_interp_V1
\end{snippet}

To build the firmware, type `make`. This should complete without incident. After the make command completes 
successfully type `make upload` to flash the firmware to the device. Please ensure the device is programmed via USB,
and that the USB cable is plugged into the programming port of the device. This is the port closest to the DC power port. 

If you get an error along the lines of `import failed, serial not found`, this is likely because \tx{arduino-mk} is using 
a different version of python than the rest of the project. To mitigate this, install pyserial for \tx{python2}:

\begin{snippet}
pip install pyserial
\end{snippet}


\subsection{Setting up the Sigilent 1020XE Oscilloscope}\label{setup-osc}

This toolchain works with the Sigilent 1020XE Oscilloscope. Any part of the
toolchain that requires the oscilloscope be set up will be annotated 
an \tx{[OSC]} tag. To use the oscilloscope, make the following connections:

\begin{enumerate}
  \item Connect the first channel to pin XXX (positive channel of
    chip[0,3,2])
  \item Connect the second channel to pin XXX (negative channel of chip
    [0,3,2]).
  \item Connect the EXT lead to pin 25 of the Arduino. The Arduino triggers the
    oscilloscope to record data.
\end{enumerate}

Connect the oscilloscope to an Ethernet port. Write down the IP address
of the oscilloscope (XXX->XXX) and make sure you can access it from your
computer. You can use the following command to test the oscilloscope is
reachable:
\begin{snippet}

\end{snippet}


\section{Configuring the Legno Toolchain}

The \legno toolchain must first be configured before it can be used. First, copy
the reference configuration:

\begin{snippet}
cp util/config_local.py util/config.py
\end{snippet}

Next, replace the following fields:

\begin{enumerate}
\item \tx{OSC_IP}: populate this field with the IP address of the oscilloscope.
  If you are not using an oscilloscope, you can use the default IP address.
\item\tx{DEVSTATE_PATH}: set this field to the directory where all the device
  data (calibration information, analog block models, etc) should be stored. We
  recommend leaving this field unchanged.
\item\tx{OUTPUT_PATH}: set this field to the directory where all the compiler
  outputs should be stored. We recommend leaving this field unchanged.
\end{enumerate}

The remaining fields are automatically populated from these three fields. We
don't recommend directly setting any of these. The \tx{ARDUINO_FILE_DESC} field
is automatically populated by looking for the proper file descriptor in the
\tx{/dev/} directory. 
\chapter{\tx{LScale} Compilation Pass}
\label{lec:lscale}
This section describes how to use the lscale compilation pass.

\section{Example: Cosine Function}

Next, we direct the \legno compiler to scale each unscaled analog device program
in the \tx{abs-circ} directory:

\begin{snippet}
  python3 legno.py --subset extended lscale cos --search \
     --model naive-max_fit --scale-circuits 1
\end{snippet}

This step of the compilation process is called the \lscale compilation pass. The
\tx{--subset} flag indicates what subset of device features to use. The
\tx{--model} argument indicates models should be used when scaling the system.
\legno supports four different modeling schemes, some of which involve delta
models. A delta model is an empirically derived model that describes the physical behavior of the
hardware. Please refer to Section~\ref{sec:delta} for information on how to use
delta models with \lscale:

\begin{itemize}
\item\tx{delta-max_fit}: Scale the system using delta models derived from block
  behavior, when the block is calibrated with the \tx{max_fit} strategy. The
  \tx{max_fit} calibration strategy maximizes the chances a delta model can be fit to
  account for deviations in block behavior.
\item\tx{delta-min_error}: Scale the system using delta models derived from
  block behavior, when the block is calibrated with the \tx{min_error} strategy.
  The \tx{min_error} calibration strategy minimizes the error of each block.
\item\tx{naive-max_fit} and \tx{naive-min_error}: Scale the system without delta
  models. With these schemes, the block behavior is as described in the
  specification. These is no behavioral difference in these models; there are
  two models for implementation reasons. 
\end{itemize}

We do not have any delta models for the chip yet, so we will use the \tx{naive-max_fit}
configuration. The \tx{naive_maxfit} model assumes each block delivers its
expected behavior with no deviations. The \tx{--scale-circuits} parameter
determines how many scaled programs to produce from each unscaled program.
Finally the \tx{--search} parameter tells \legno to search for the scaling
transform that produces the best signal-to-noise ratio.

For the \tx{cos} dynamical system with the \tx{extended} set of features, the
resulting scaled programs are written to the following directory:

\begin{snippet}
  legno-compiler/outputs/legno/extended/cos/lscale-adp/
\end{snippet}

The \lscale execution presented above produces exactly one scaled analog device program: 

\begin{snippet}
  cos_g0x0_s0_ngd3.00a3.77v1.77c97.00_obsfast.adp
\end{snippet}

Since the analog device program is not human readable, the compiler
also produces graphs that visually depict the circuit each program implements.
These graphs are stored in the following directory:

\begin{snippet}
  legno-compiler/outputs/legno/extended/cos/lscale-adp/
\end{snippet}
\chapter{Compilation with Delta Models}

A \textit{delta model} is an empirically derived model that describes a block's
behavior. This empirical model can be used by the compiler to compensate for any
manufacturing deviations found in the blocks.

\section{Compiling the Cosine Program with Delta Models}

Delta models are elicited from the chip on-demand. The general workflow for
extracting delta models from the device is presented below:

\begin{enumerate}
\item \textbf{wishlist generation}: We ask the \legno compiler to generate a list of configured
  blocks to calibrate and profile, in the form of a grendel script. The blocks
  on this list appear in the provided program, but are missing delta models.
\item \textbf{calibration and profiling}: We use the \grendel runtime to
  calibrate and profile the configured blocks on the wishlist. The calibration
  and profiling information are stored in \tx{state.db}
\item \textbf{model inference}: We use the model builder to infer models from
  the profiling information. The inferred models are stored in \tx{model.db}
\item \textbf{compilation with delta models}: We invoke the reexecute the \legno
  compiler with the same arguments as in the first step. This time, \legno
  is able to find all the delta models it needs, and successfully compile the circuit.
  
\end{enumerate}


\subsection{Generating the Delta Model Wishlist}

First, we need to request a list of configured blocks to calibrate and profile.
The following command asks the compiler to attempt to scale the circuits for the
cosine benchmark using delta models (the \tx{delta-max_fit} model uses delta models): 

\begin{snippet}
  python3 legno.py --subset extended lscale cos --search \
     --model delta-max_fit --scale-circuits 1
   \end{snippet}

This command should fail because \legno does not have the delta models
necessary to compile the circuit. You should see output that looks something
like this:

\begin{snippet}
  ...
  scale mode dne: CalibrateObjective.MAX_FIT fanout[(HDACv2,0,3,0,0)].in cm=('pos', 'neg', 'pos') scm=high handle=None
  scale mode dne: CalibrateObjective.MAX_FIT fanout[(HDACv2,0,3,0,0)].out1 cm=('pos', 'neg', 'pos') scm=high handle=None
  scale mode dne: CalibrateObjective.MAX_FIT fanout[(HDACv2,0,3,0,0)].out2 cm=('pos', 'neg', 'pos') scm=high handle=None
scale mode dne: CalibrateObjective.MAX_FIT fanout[(HDACv2,0,3,0,0)].out0 cm=('pos', 'neg', 'pos') scm=high handle=None
no valid scale modes: fanout[(HDACv2,0,3,0,0)]
scale mode dne: CalibrateObjective.MAX_FIT ext_chip_out[(HDACv2,0,3,2,0)].in cm=* scm=* handle=None
scale mode dne: CalibrateObjective.MAX_FIT ext_chip_out[(HDACv2,0,3,2,0)].out cm=* scm=* handle=None
no valid scale modes: ext_chip_out[(HDACv2,0,3,2,0)]
scale mode dne: CalibrateObjective.MAX_FIT tile_out[(HDACv2,0,3,0,0)].in cm=* scm=* handle=None
scale mode dne: CalibrateObjective.MAX_FIT tile_out[(HDACv2,0,3,0,0)].out cm=* scm=* handle=None
no valid scale modes: tile_out[(HDACv2,0,3,0,0)]
  ...
  ...
  File "/Users/sachour/Documents/git/research/legno-compiler/compiler/lscale.py", line 113, in scale
    raise Exception("must calibrate components")
Exception: must calibrate components
\end{snippet}

When this happens, the compiler also writes out a list of configured blocks to
calibrate and profile to the following location. Note that if we're using
\tx{delta-min_error}, the file will be named \tx{min_error.grendel} instead of \tx{max_fit.grendel}:

\begin{snippet}
  device-state/calibrate/max_fit.grendel
\end{snippet}

The file contains a list of block use commands and should look something like
this:

\begin{snippet}
use_fanout 0 3 0 0  sgn + - + rng m two
use_fanout 0 3 0 0  sgn + - + rng h two
use_mult 0 3 0 0 val 0.0 rng m m
use_mult 0 3 0 0 val 0.0 rng h h
use_mult 0 3 0 0 val 0.0 rng h m
use_mult 0 3 0 0 val 0.0 rng m h
\end{snippet}


\subsection{Calibrating and Profiling Blocks on Wishlist}

After acquiring the list, we can calibrate and profile the blocks. We calibrate
the blocks using the following command:

\begin{snippet}
  python3 -u grendel.py calibrate --no-oscilloscope --calib-obj max_fit \
     device-state/calibrate/max_fit.grendel

\end{snippet}

This will update \tx{device-state/state.db} to contain both the calibration and
profiling information for each block. Note that changing \tx{--calib-obj} argument to
\tx{min_error} will calibrate blocks using the \tx{min_error} objective
function. This step takes a long time, so get a coffee.


\textbf{Profiling}: The blocks can be calibrated after they are profiled. The
profiling command is shown below:

\begin{snippet}
  python3 -u grendel.py profile --no-oscilloscope --calib-obj max_fit \
     device-state/calibrate/max_fit.grendel
\end{snippet}


\textit{An Aside}: The profiling information can be exported to a collection of
\tx{.json} files for further analysis with the following command:

\begin{snippet}
  python3 grendel.py dump
\end{snippet}

This command writes the extracted datasets to the following directory:

\begin{snippet}
outputs/datasets
\end{snippet}


\subsection {Inferring Models from Profile Information}

After the blocks have been profiled, we can infer the delta models. Execute the
following command to infer the delta models from the profiling information:

\begin{snippet}
  python3 model_builder.py infer --calib-obj max_fit --visualize
\end{snippet}

The \tx{--visualize} flag writes visualizations of the block
error and model error to \tx{device-state/models}. The \tx{--calib-obj} flag
tells the model inference routine tells us which set of profiling information to
use.

\subsection{Compiling with the Delta Models}

\noindent To use the inferred delta models, execute the following command:
 
\begin{snippet}
  python3 legno.py --subset extended lscale cos --search \
     --model delta-max_fit --scale-circuits 1
\end{snippet}

\noindent This should now produce scaled circuits using scaled circuits that take into
account the gains and time constant variations of the blocks. This command will
write a new scaled program to the following directory:

\begin{snippet}
  legno-compiler/outputs/legno/extended/cos/lscale-adp/
\end{snippet}

\noindent The scaled program should have a name similar to the one below. Notice
the tag in the file begins with \tx{d} instead of \tx{n}, this indicates it was compiled using a delta model:

\begin{snippet}
  cos_g0x0_s0_dgd3.00a3.77v1.77c97.00_obsfast.adp
\end{snippet}

\noindent Next, we can generate a \grendel script from the scaled circuit. The
source generation routine will adjust the constants and functions written to
the lookup tables to account for any biases. Execute the following command to do so:

\begin{snippet}
  python3 legno.py --subset extended srcgen cos --hwenv osc --trials 1
\end{snippet}

\noindent This will write a \grendel script to the following directory:

\begin{snippet}
  outputs/legno/extended/cos/grendel
\end{snippet}

\noindent The \grendel script should have a name similar to the one below:

\begin{snippet}
  cos_g0x0_s0_dgd3.00a3.77v1.77c97.00_obsfast_t20_osc.grendel
\end{snippet}

We will refer to the this file as \tx{<file>.grendel} for the remainder
of this section.

\subsection{Executing and Analyzing the }




Then repeat the source generation and execution commands described above.

XXX srcgen, run, scan, analyze XXX
\chapter{Legno Quickstart}

The following quickstart walks through compiling a dynamical system program that
implements the cosine (\tx{cos}) function. This section 


\section{Compiling the \tx{cos} Program}

The following section describes how to compile the cosine program. We use the
\tx{legno_runner.py} script to compile the program. This script executes all
the necessary compiler passes (by invoking \tx{legno.py} multiple times)
to generate \tx{grendel} scripts, using the parameters defined in
\tx{configs/quickstart.cfg}. The cosine program is compiled by executing the following
command from the \tx{legno-compiler} directory:

\begin{snippet}
  python3 legno_runner.py --config configs/quickstart.cfg --hwenv osc  \
    cos --lgraph
\end{snippet}

This command executes all the necessary compiler passes to generate
\tx{.grendel} scripts, using the parameters defined in
\tx{configs/quickstart.cfg}. The grendel script can then be executed on the
analog device.

\begin{comment}
\noindent\textbf{Without Oscilloscope}: If you would like to compile the cosine
program without any oscilloscope support, change the \tx{--hwenv osc} argument to
\tx{--hwenv noosc}. 
\end{comment}

\subsection{The \tx{quickstart.cfg} File}

The \tx{quickstart.cfg} file is a \tx{json} dictionary with the following fields:

\begin{snippet}
{
  "subset":"extended",
  "model":"naive-max_fit",
  "n-lgraph":1,
  "n-lscale":1,
  "max-freq":80
}
\end{snippet}

This configuration file specifies the parameters to provide to each compilation
pass in the compilation process. We describe the fields below:

\begin{itemize}
\item\textbf{subset}: The subset argument is passed in at each step of the
  compilation process. It dictates what subset of analog device features the
  compiler should use. The \tx{extended} subset allows the compiler to use
  medium ([-2,2] uA) and high ([-20,20] uA) modes. It is the broadest set of
  tested features on the device. 
\item\textbf{model}: The model argument tells the circuit scaling pass
  (\tx{lscale}, Section~\ref{sec:lscale}) how to model the behavior of the
  analog blocks. The \tx{naive-max_fit} model tells the circuit scaling pass
  to assume the block behavior adheres to the behavior described in the
  hardware specification (\tx{naive}). It also tells the compiler the blocks
  will be calibrated using the \tx{max_fit} calibration objective (see Section
  XXX below).
\item\textbf{n-lgraph}: The number of unscaled circuits the graph synthesis
  pass should generate (\tx{lgraph}, Section~\ref{sec:lgraph})
\item\textbf{n-lscale}: The number of scaled circuits to produce per
  unscaled circuit. This is provided to the circuit scaling pass
  (\tx{lscale}, Section~\ref{sec:lscale}).
\item\textbf{max-freq}: The maximum allowable speed of the simulation.
  This is provided to the circuit scaling pass (\tx{lscale}, Section~\ref{sec:lscale}).
\end{itemize}


\subsection{Inspecting the Compilation Outputs}

The \tx{legno_runner.py} script produces unscaled circuits, scaled circuits and
grendel scripts. All compilation and execution outputs for the cosine program is
stored in the following directory:

\begin{snippet}
  legno-compiler/outputs/legno/extended/cos/
\end{snippet}

We call this the \textit{program directory}. The following sub-directories in the
program directory contain compilation outputs:

\begin{itemize}
\item\tx{lgraph-adp} and \tx{lgraph-diag}: These directories contain the
  unscaled analog device programs and associated diagrams. The diagrams are visual
  representations of the programs that are useful for debugging.
\item\tx{lscale-adp} and \tx{lscale-diag}: These directories contain the
  scaled analog device programs and associated diagrams. 
\item \tx{grendel}: This directory contains the compiled \tx{.grendel} scripts
  generated by the \tx{srcgen} pass of the \legno compiler.
\item\tx{times}: this directory contains the runtime information for each
  pass.
\end{itemize}

We should see one \tx{.adp} file in the \tx{lscale-adp} and \tx{lgraph-adp}
directories, and one \tx{.png}/\tx{.dot} file in the \tx{lgraph-diag} and
\tx{lscale-diag} directories. There should be one \tx{.grendel} file in the
\tx{grendel} directory that should have a name similar to the one below:

\begin{snippet}
  cos_g0x0_s0_ngd3.00a12.28v1.77c97.00b80.00k_obsfast_t20_osc.grendel
\end{snippet}

For brevity, we will refer to this file as \tx{<file>.grendel} for the rest of
the quickstart section. 

\section{Running the Cosine Grendel Script}

The grendel runtime (\tx{grendel.py}) executes \tx{.grendel} scripts generated
by the \legno compiler. Before executing the script, we must calibrate any
uncalibrated blocks. To do so, execute the following command from the
\tx{legno-compiler} directory:


\begin{snippet}
python3 grendel.py calibrate --calib-obj max_fit \
  outputs/legno/extended/cos/grendel/<file>.grendel 
\end{snippet}

The \tx{--calib-obj} argument specifies which calibration objective to use.
We use the \tx{max_fit} calibration objective.
We can run the benchmark after all the blocks have been calibrated. Execute the
following command to execute the program an the analog computer:

\begin{snippet}
python3 grendel.py run --calib-obj max_fit \
  outputs/legno/extended/cos/grendel/<file>.grendel 
\end{snippet}

If the script executed correctly, you should see a \tx{.json} file appear in
the \tx{out-waveforms} directory. This file is the waveform downloaded from the
oscilloscope:

\begin{snippet}
legno-compiler/outputs/legno/extended/cos/out-waveform/
\end{snippet}

\subsection{Execution without the Sigilent 1020XE Oscilloscope}:

If you are not using the Sigilent 1020XE oscilloscope, include the
\tx{--no-oscilloscope} flag to \tx{grendel.py} commands to prevent the runtime
from attempting to communicate with the measurement device. If you would like to
use an unsupported oscilloscope, configure the voltage and time divisions on the
oscilloscope using the voltage and time ranges in the \tx{<file>.grendel} file
(see Section \ref{sec:grendel}) and set it to wait for an edge trigger.

If the script executed correctly, you should see a cosine waveform on your
oscilloscope. Note that you will not see a \tx{.json} file in the \tx{out-waveforms}
directory, because the runtime cannot communicate with an unsupported oscilloscope.


\section{Analyzing Oscilloscope Waveform Data [OSC]}

The \tx{exp_driver.py} tool is used to manage and analyze oscilloscope outputs.
We execute the following command from the \tx{legno-compiler} directory to
discover all the grendel scripts and oscilloscope waveforms in the \tx{output}
directory:

\begin{snippet}
  python3 exp_driver.py scan
\end{snippet}

\noindent The discovered waveforms can then be analyzed with the following command.
Currently, \tx{exp_driver.py} is able to automatically compare waveforms to digital
simulations:

\begin{snippet}
    python3 exp_driver.py analyze
\end{snippet}

The oscilloscope waveforms and visual analysis results are written to the
\tx{plots} directory in the program directory:

\begin{snippet}
  outputs/legno/extended/cos/plots
\end{snippet}


